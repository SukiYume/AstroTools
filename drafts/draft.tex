\documentclass[11pt,letterpaper]{article}

\newenvironment{proof}{\noindent{\bf Proof:}}{\qed\bigskip}

\newtheorem{theorem}{Theorem}
\newtheorem{corollary}{Corollary}
\newtheorem{lemma}{Lemma} 
\newtheorem{claim}{Claim}
\newtheorem{fact}{Fact}
\newtheorem{definition}{Definition}
\newtheorem{assumption}{Assumption}
\newtheorem{observation}{Observation}
\newtheorem{example}{Example}
\newcommand{\qed}{\rule{7pt}{7pt}}

\newcommand{\assignment}[5]{
\thispagestyle{plain} 
\newpage
\setcounter{page}{1}
\noindent
\begin{center}
\framebox{ \vbox{ \hbox to 6.28in
{\bf \textsf{ #1 \hfill #2 }}
\vspace{4mm}
\hbox to 6.28in
{\hspace{2.5in}\Large\mbox{\bf \textsf{Problem Set #3}}}
\vspace{4mm}
\hbox to 6.28in
{{\it Handed Out: #4 \hfill Due: #5}}
}}
\end{center}
}

\newcommand{\normaldoc}[5]{
\thispagestyle{plain} 
\newpage
\setcounter{page}{1}
\noindent
\begin{center}
\framebox{ \vbox{ \hbox to 6.28in
{\bf \textsf{ #1 \hfill #2 }}
\vspace{4mm}
\hbox to 6.28in
{\hfill\Large\mbox{\bf \textsf{#3}}\hfill}
\vspace{4mm}
\hbox to 6.28in
{#4 \hfill {Completed Date:  \textit{#5} }}
}}
\end{center}
\markright{#4}
}

\newcommand{\project}[6]{
\thispagestyle{plain} 
\newpage
\setcounter{page}{1}
\noindent
\begin{center}
\framebox{ \vbox{ \hbox to 6.28in
{\bf \textsf{ #1 \hfill #2 }}
\vspace{4mm}
\hbox to 6.28in
{\hfill\Large\mbox{\bf \textsf{Project #3: #6}}\hfill}
\vspace{4mm}
\hbox to 6.28in
{#4 \hfill {\textit{#5} }}
}}
\end{center}
\markright{#4}
}

\newcommand{\solution}[5]{
\thispagestyle{plain} 
\newpage
\setcounter{page}{1}
\noindent
\begin{center}
\framebox{ \vbox{ \hbox to 6.28in
{\bf \textsf{ #1 \hfill #2 }}
\vspace{4mm}
\hbox to 6.28in
{\hspace{2.5in}\Large\mbox{\bf \textsf{Problem Set #3}}}
\vspace{4mm}
\hbox to 6.28in
{#4 \hfill {Completed Date: \textit{#5} }}
}}
\end{center}
\markright{#4}
}

\newenvironment{algorithm}
{\begin{center}
\begin{tabular}{|l|}
\hline
\begin{minipage}{1in}
\begin{tabbing}
\quad\=\qquad\=\qquad\=\qquad\=\qquad\=\qquad\=\qquad\=\kill}
{\end{tabbing}
\end{minipage} \\
\hline
\end{tabular}
\end{center}}

\def\Comment#1{\textsf{\textsl{$\langle\!\langle$#1\/$\rangle\!\rangle$}}}



\usepackage{graphicx,amssymb,amsmath,enumerate}
\usepackage{subfigure}
\usepackage{hyperref}
\usepackage{courier}
\usepackage{color}
\usepackage{listings}
\usepackage{algorithm}
\usepackage[noend]{algpseudocode}
\definecolor{dkgreen}{rgb}{0,0.6,0}
\definecolor{gray}{rgb}{0.5,0.5,0.5}
\lstset{language=C++,
	frame=lines,
   basicstyle=\ttfamily\fontsize{8}{12}\selectfont,
   keywordstyle=\color{blue},
   commentstyle=\color{red},
   stringstyle=\color{dkgreen},
   numbers=left,
   numberstyle=\tiny\color{gray},
   stepnumber=1,
   numbersep=10pt,
   backgroundcolor=\color{white},
   tabsize=2,
   showspaces=false,
   showstringspaces=false,
   lineskip=-3.5pt }
\oddsidemargin 0in
\evensidemargin 0in
\textwidth 6.5in
\topmargin -0.5in
\textheight 9.0in

\renewcommand{\algorithmicrequire}{\textbf{Input:}}
\renewcommand{\algorithmicensure}{\textbf{Output:}}

\begin{document}

\title{Notes on Quadrotor}
\maketitle

\pagestyle{myheadings}  % Leave this command alone

\section{Mathematic}
Assume a star with Equatorial coordinate $(\delta, \alpha)$, the latitude is $\phi_0$, the camera is pointing
at $(a, A)$ in Horizontal coordinate. 

The $x-y-z$ coordinate in Equatorial frame is
\begin{equation}
    \boldsymbol{p}_e = (\cos\delta \cos(\alpha + \omega t),\; \cos\delta \sin(\alpha + \omega t),\; \sin\delta)
\end{equation}

and in Horizontal frame is
\begin{equation}
    \boldsymbol{p}_h = \boldsymbol{R}_x\left(\phi_0-\frac{\pi}{2}\right)\boldsymbol{p}_e
\end{equation}

and in camera frame is
\begin{equation}
    \boldsymbol{p}_c = \boldsymbol{R}_x\left(\frac{\pi}{2}-a\right)\boldsymbol{R}_z(A)\boldsymbol{p}_h
\end{equation}

and every element can be expressed as
\begin{equation}
\begin{split}
    x =& \cos A \cos \delta  \cos (\alpha +\omega t)-\sin A (\cos \delta  \sin
        \phi_0 \sin (\alpha +\omega t)+\sin \delta  \cos \phi_0) \\
    y=&\sin \delta  \cos(A \cos \phi_0 \sin a - \cos a \sin \phi_0) + 
 \cos \delta (\cos(\alpha + \omega t) \sin a \sin A +  \\
 &\cos a \cos \phi_0 + 
 \cos A \sin a \sin \phi_0) \sin(\alpha + \omega t))\\
    z =&\cos \delta 
       \sin (\alpha +\omega t) (\cos a \cos A \sin \phi_0-\sin a \cos
       \phi_0)+\cos a \sin A \cos \delta  \cos (\alpha +\omega t)+ \\
       &\cos a \cos A \sin \delta  \cos \phi_0+\sin a \sin \delta  \sin \phi_0
\end{split}
\end{equation}

\end{document}
